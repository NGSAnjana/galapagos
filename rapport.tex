\documentclass{article}
\usepackage[latin1]{inputenc}

\pagestyle{plain}

\title{Eksamensproject i Object-Orienteret Programmering \& Design}

\author{Martin Dybdal, Mikkel Abrahamsen, Sune Precht Reeh, Troels Henriksen}

\date{Januar 2007}
\begin{document}
\maketitle

\section{Forfatterinformation}

Mikkel Abrahamsen \\
\noindent
Jagtvej 120,3,-346 \\
\noindent
2200 K�benhavn N \\
\noindent
Eksamensnummer: 37

\section{Indledning}

\section{Programdesign}

\subsection{Model}

\subsubsection{\tt World}

Den fundamentale datastruktur, der ligger til grund for hele
simulationens  verden, er den passende navngivne klasse {\tt World}.
Denne klasse giver i sin enkelthed adgang til et todimensionelt array
af {\tt World.Place} objekter. Disse {\tt World.Place}s skaber
verdens torusform, som kr�vet i opgaven, idet de har metoder til at
finde nabo-{\tt World.Place}s. Et af disse objekter, placeret p� en
kant, vil give de passende objekter p� den modsatte kant, n�r der
bliver spurgt om sine naboer. Et {\tt World.Place} objekt kan ogs�
indeholde et andet objekt - dets {\em element}. I vores simulation er
finke-objekterne placeret som elementer i {\tt World.Place} objekter,
men da man kan forestille sig at der sidenhen kan v�re behov for at
lave {\tt World}-strukturer med andre typer objekter end finker, har
vi valgt at benytte generics til {\tt World}-klassen.

Metoderne {\tt World.Place.filledNeighbors} og {\tt
  World.Place.emptyNeighbors} bruges til at finde de naboceller der
  hhv. indeholder elementer eller er tommer. De skal benyttes n�r
  finker skal f� afkom og m�de andre finker.

Under simulationen er der ydermere endnu en interessant facilitet, vi
ofte har behov for at udf�re p� vores verden - n�r der skal arrangeres
m�der og skabes afkom, skal vi iterere igennem alle finker i verden,
men for ikke skabe favoritter, skal dette helst ske i pseudotilf�ldig
r�kkef�lge. Ydermere �nsker vi et iteratorbaseret interface for at
g�re koden s� elegant som muligt. En �benlys implementation er at lade
{\tt World} klone dets underliggende lager (en enkeltdimensional {\tt
  ArrayList}), kalde {\tt Collections.shuffle} p� det nye array, og
returnere en iterator. Dette ville resultere i allokeringen af et nyt
array for hver gang vi har behov for at iterere igennem elementerne i
{\tt World} i pseudotilf�ldig r�kkef�lge - en �benlys ineffektiv
implementation. Vi har dog - ved at observere at {\tt
  World.Place}-objekterne i {\tt World} aldrig �ndrer sig, kun deres
elementer - fundet p� en mere effektiv implementation, hvor vi i {\tt
  World}, sidel�bende med det almindelige lager, gemmer et alternativt
lager, der indeholder de samme {\tt World.Place} objekter som det
prim�re, men i en anden r�kkef�lge. Idet det er de samme {\tt
  World.Place}-objekter der bliver modificeret n�r der tilf�jes og
fjernes finker, vil dette array altid v�re opdateret mht. indhold, og
vi kan implementere tilf�ldig iteration blot ved at kalde {\tt
  Collections.shuffle} p� det og returnere en iterator. Derved slipper
vi for at skulle klone vores underliggende array hver gang vi har
behov for pseudotilf�ldig iteration.

\subsection{Brugerinterface}

\section{Konklusion}

\end{document}
